\begin{abstract}
    
Raindrops adhering to a camera lens blur texture, distort colour and thwart downstream computer-vision tasks that underpin autonomous driving, sports analytics and surveillance. This project aims to develop methods to remove raindrops from an image by utilising the power of deep learning models. The models explored have two main parts, a raindrop mask generating network, which identifies the regions in an image where raindrops are present, and a raindrop removal network, which recovers a raindrop free image. Extensive datasets are needed to train the deep learning models effectively. So a main part of the project was the collection of data from various environments and conditions where raindrops could adhere to camera lenses. In total, 320 images were captured, with 80 different backgrounds, each background consisting of 4 images, a clean image, and 3 varying levels of raindrop degradation. The state-of-the-art model is also recreated (model 1) with a novel introduction into the training process. To compare, the original state-of-the-art is also retrained and retested (model 2). A final model with the same underlying architecture, but that wasn't trained in an end-to-end fashion is also used as a benchmark for comparison (model 3). The quantitative metrics used to evaluate these models are the average peak signal to noise ratio (PSNR in dB), average structural similarity index metric (SSIM). All models achieve their maximum results on test set A, where model 1 achieves 27.1950 dB/0.8666, and models 2 and 3 achieve 26.3685 dB/0.8307, and 25.8567/0.8495, respectively. The novel implementation of the state-of-the-art allows it to outperform the other iterations of the model. The proposed novel model uses the same framework, with a mask generation network and a rain removal network. The mask generation network leverages semantic segmentation with uncertainty to create a mask generator robust to noise and uncertainty. My proposed novel model in the paper achieves results of 26.4808/0.8605. My novel model outperforms the original state-of-the-art when benchmarked on the same training and test set, and is only outperformed by the modified state-of-the-art from my contributions. 

\end{abstract}